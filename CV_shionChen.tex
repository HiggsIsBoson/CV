\documentclass[12pt]{article}
%\documentclass{moderncv}
%\moderncvtheme[blue]{classic}
%\firstname{Shion}
%\familyname{Chen}
%\usepackage{multibib}


\setlength{\topmargin}{-1.5cm}
\setlength{\oddsidemargin}{-0.3cm}
\setlength{\evensidemargin}{-0.3cm}
\setlength{\textwidth}{16.5cm}
\setlength{\textheight}{23cm}
\usepackage[utf8]{inputenc}
\usepackage{graphicx}
\usepackage{float}
\usepackage{slashed}
\usepackage{caption}
\usepackage{url, braket, setspace}
\usepackage{amsmath,amsthm,amssymb,bm}
%\usepackage{hyperref}

\setlength{\abovedisplayskip}{4pt} % upper margin
\setlength{\belowdisplayskip}{8pt}% lower margin


%%%%%%%%%%%%%%% footer
%\usepackage{lipsum} % for filler text
%\usepackage{fancyhdr}
%\pagestyle{fancy}
%\fancyhead{} % clear all header fields
%\renewcommand{\headrulewidth}{0pt} % no line in header area
%\fancyfoot{} % clear all footer fields
%\fancyfoot[LE,RO]{\thepage}           % page number in "outer" position of footer line
%\fancyfoot[RE,LO]{Shion Chen} % other info in "inner" position of footer line



%%%%%%%% reference
\renewcommand{\refname}{Full List of Publication \hrule height 0.5mm depth 1mm width 100mm}
%\usepackage[sorting=none, backend=biber]{biblatex} % load the package
%\addbibresource{publication/testBI} 
%\addbibresource{publication/ILC_calorimeter.bib} 
%\addbibresource{publication/ATLAS} 

%%%%%%%% section
\newcommand{\Section}[1]{\section*{#1}
\addcontentsline{toc}{section}{#1}}
\newcommand{\Subsection}[1]{\subsection*{#1}
\addcontentsline{toc}{subsection}{#1}}
\newcommand{\Subsubsection}[1]{\subsubsection*{#1}
\addcontentsline{toc}{subsubsection}{#1}}

\usepackage{titlesec}
\titleformat{\section}
  {\normalfont\sffamily\Large\bfseries  }
  {\thesection }{1em}{}
\titleformat{\subsection}
  {\normalfont\sffamily \large \bfseries }
  {\thesection}{1em}{}
\titleformat{\subsubsection}
  {\normalfont\sffamily \bfseries }
  {\thesection}{1em}{}

%%%% Alias %%%%%
\newcommand{\beq}{\begin{equation}}
\newcommand{\eeq}{\end{equation}}
\newcommand{\colv}[1]{\left( \begin{array}{c} #1 \end{array} \right)}
\newcommand{\colvf}[1]{\left( \begin{array}{cccc } #1 \end{array} \right)}
\newcommand{\tab}[4]{\begin{table}[h] \begin{center} \caption{#3} \begin{tabular}{#1} #2 \end{tabular}  \label{#4} \end{center}   \end{table}}
\newcommand{\tabnocap}[3]{\begin{table}[h] \begin{center} \caption*{} \begin{tabular}{#1} #2 \end{tabular}  \label{#3} \end{center}   \end{table}}
%
\def \mm#1{\mathrm{#1}}
% fig 
\newcommand{\fig}[4][width=15cm]{
\begin{figure}[H]
 \centering
 \includegraphics[width=#1mm]{#2.eps}
 \caption{#3}
 \label{#4}
 \end{figure}} 
%
% Unit
\newcommand{\gev}{\mathrm{GeV}}
\newcommand{\mev}{\mathrm{MeV}}
%
%%%%%%%%%%%%
\begin{document}
%\lipsum[1-20] % generate about 4 pages of filler text

\begin{center}
  \large{\textbf{Shion Chen  - {\it Curriculum Vitae}   }}
\end{center}

\tabnocap{l l} {
Date of birth: & 17 June 1989 \\
Nationality: & Japan \\
Address: & Department of Physics and Astronomy, \\
               & University of Pennsylvania. 209 S. 33rd 19104 PA USA. \\
email: &  cshion@sas.upenn.edu \\
phone: & +1-267-576-9389 / +41-75411-3595 (CERN)
}{}

\Section{Education and Training \hrule height 0.5mm depth 1mm width 100mm}
\begin{description}
	 \item[\underline{2014-2017}]   \mbox{}\\
	 	Doctoral student, The University of Tokyo, Japan. JPS fellowship. \\
      	        Thesis: ``Search for Gluinos using Final States with One Isolated Lepton in the LHC-ATLAS Experiment''. \\
                Advisor: Sachio Komamiya, Shoji Asai.
	 \item[\underline{2012-2014}]  	\mbox{}\\
	 	Master's degree in Physics, The University of Tokyo, Japan.  \\ 
      	        Thesis: ``Test of Bell's Inequality and Entanglement Measurement in Collider Experiments". \\
                Advisor: Sachio Komamiya.
\end{description}


\Section{Work Experience \hrule height 0.5mm depth 1mm width 100mm}
\begin{description}
	 \item[Post-doctoral fellow] (University of Pennsylvania, Sep 2017-) \mbox{} \\
               Main project: Searches for Super-symmetric particle, new lepton tagger development and the operation of the Transition Radiation Tracker in ATLAS experiment. \\
	       Advisor: Brig Williams.               
\end{description}
\clearpage


%%%%%%%%%%%%%%%%%%%%%%% Pulication %%%%%%%%%%%%%%%%%%%%%%%
\Section{Selected publications \hrule height 0.5mm depth 1mm width 100mm} 
\begin{enumerate}
	\item \underline{\textbf{S. Chen}}, Y. Nakaguchi and S. Komamiya \\
			``Testing Bell's Inequality using Charmonium Decays'', 
			Prog. Theor. Exp. Phys. (2013) 063A01. \label{Pub::BI2013_PTEP}
%
	\item Robustness of a SiECAL used in Particle Flow Reconstruction, C.Kozakai, \underline{\textbf{S.Chen}} et.al. (1404.0124)	
		International Workshop on Future Linear Colliders (LCWS13) Tokyo, Japan, November 11-15, 2013. \label{Pub::ECAL_LCWS2013}
%
        \item  M. S. Amjad, \dots \underline{\textbf{S.Chen}} et al. Beam test performance of the SKIROC2 ASIC. Nucl. Instrum. Meth., A778:78–84, 2015. \label{Pub:SKIROC2015}
%
	\item ATLAS Collabolation  \dots \underline{\textbf{S. Chen}} \dots \\
		   ``Search for gluinos in events with an isolated lepton, jets and missing transverse momentum at $\sqrt{s}= 13$ TeV with the ATLAS detector."
	             Eur. Phys. J., C76(10):565, 2016. \label{Pub::ATLPaper_Incl1L_2015}  
%
  	\item ATLAS Collabolation \dots \underline{\textbf{S. Chen}} \dots \\ 
		   ``Search for squarks and gluinos in events with an isolated lepton, jets and missing transverse momentum at $\sqrt{s} = 13$ TeV with the ATLAS detector."
	            ATLAS-CONF-2016-054, 2016. \label{Pub::ATLCONF_Incl1L_ICHEP2016} 
%
	\item ATLAS Collabolation \dots \underline{\textbf{S. Chen}} \dots \\
		   ``Search for supersymmetry with two and three leptons and missing transverse momentum in the final state at $\sqrt{s}=13$ TeV with the ATLAS detector."
		    ATLAS-CONF-2016-096, 2016 \label{Pub::ATLCONF_EW_SEARCH2016}
%
	\item ATLAS Collabolation \dots \underline{\textbf{S. Chen}} \dots \\
		   ``Search for squarks and gluinos in events with an isolated lepton, jets and missing transverse momentum at $\sqrt{s}=13$ TeV with the ATLAS detector.''
		    Accepted by Physics Review D, 2017 	\label{Pub::ATLPaper_Incl1L_2017}
%
	\item ATLAS Collabolation \dots \underline{\textbf{S. Chen}} \dots \\
		   ``Searches for electroweak production of supersymmetric particles with compressed mass spectra in $\sqrt{s}=13$~TeV $pp$ collisions with the ATLAS detector''
		    ATLAS-CONF-2019-014, 2019 \\ 	\label{Pub::ATLCONF_EWCompressed_LHCP2019}

\end{enumerate}


%%%%%%%%%%%%%%%%%%%%%%% Talks at Confereces %%%%%%%%%%%%%%%%%%%%%%%
\Section{Selected Talks at Conferences \hrule height 0.5mm depth 1mm width 100mm}
\begin{enumerate}
\item  ``Testing Quantum Mechanics in Collider Experiments'', International School of Sub-nuclear Physics 2015, July 2015, Erice, Italy. \label{Talk:ERICE2015}
%
\item ``Test beam analysis of SiW ECAL physics prototype in 2011 FNAL''. CALICE Collaboration Meeting, LAPP, Annecy, France, 2013. \label{Talk:CALICE2013_FNALTB}
%\item Simulation Study on SiW ECAL optimization. In CALICE Collabo- ration Meeting, LAPP, Annecy, France., 2013.
\item  ``Test of Power Pulsing with the HBU-LED System''. CALICE Collaboration Meeting, LAPP, Annecy, France, 2013. \label{Talk:CALICE2013_PPHBU}
%
\item  ``A new background suppression technique for LHC-ATLAS Run2 Electroweak Gaugino search'',  Japan physics society meeting, September 2015, Osaka Japan.
%\item  ``Search for gluinos in events with a isolated lepton, jets and missing transverse momentum at LHC-ATLAS experiment (2)",  Japan physics society meeting, March 2016, Sendai Japan.
\item  ``Search for super-symmetric particles in one lepton final state in the LHC-ATLAS experiment Run2.'',  Japan physics society meeting, September 2016, Miyazaki Japan.
\item  ``Searches for squarks and gluinos in final states with an isolated charged lepton, jets, and missing transverse momentum with the ATLAS detector'', Epiphany 2017, January 2017, Krakow, Poland. 
\item  ``Electroweak Production SUSY Searches in the ATLAS Experiment'', LHCP 2018, June 2018, Bolonga, Italy.  \\
\end{enumerate}


%%%%%%%%%%%%%%%%%%%%%%% Invited seminars %%%%%%%%%%%%%%%%%%%%%%%
\Section{Invited Seminar Talks \hrule height 0.5mm depth 1mm width 100mm}
\begin{enumerate}
\item``Test of Bell's Inequality in the Charm Factory", Shion Chen, Open Seminar talk, 2 December 2013, IHEP, China.
\item ``Testing Bell's Inequality using momentum-entangled states in Collider Experiments", Shion Chen, KEK theory seminar, 16 December 2013, KEK, Japan.
\item ``Search for electroweak and long-lived SUSY signatures at LHC Run2.", Workshop for Tera-scale physics, 23 December 2015 Tokyo Tech., Japan. \\
\end{enumerate}

%%%%%%%%%%%%%%%%%%%%%%% Grants & Awards %%%%%%%%%%%%%%%%%%%%%%%
\Section{Research Grants and Awards \hrule height 0.5mm depth 1mm width 100mm}
\begin{enumerate}
\item ``Best Science Secretariat Award", International School of Sub-nuclear Physics 2015 (July 2015, Erice, Italy).
\item CERN relay race, 2nd place, as member of Tokyo HipHoppers  (21 May 2015 CERN, Switzerland).
%\item CERN relay race, 3rd place, as member of Tokyo HipHoppers (19 May 2016 CERN, Switzerland).
\item  Research Fellow (DC1), Japan Society for the Promotion of Science (April 2014 - 2017).
\item  Competitively selected summer student project in DESY (2013). \\
\end{enumerate}

%%%%%%%%%%%%%%%%%%%%%%% Languages %%%%%%%%%%%%%%%%%%%%%%%
\Section{Languages \hrule height 0.5mm depth 1mm width 80mm}
Japanese (native), Chinese (native), English (good), French (poor), Brasilian portuguese (poor).

\clearpage 

\Section{Research Summary \hrule height 0.5mm depth 1mm width 100mm}
My research career has been focused on the investigation of elementary particles and the fundamental rules lurking behind them. 
During 5 years as a graduate student at the University of Tokyo, I have been working on various aspects of particle physics:
\begin{itemize}		
\item theoretical analysis of testing quantum nature in collider experiments
\item development of calorimeters for future lepton colliders
\item search for supersymmetry in ATLAS at the energy frontier proton-proton collider (Large Hadron Collider; LHC) at CERN. \\
\end{itemize}	

%%%%%%%%%%%%%%%%%%%%%%%%%%%%%% ATLAS %%%%%%%%%%%%%%%%%%%%%%%%%%%%%%%
\Subsection{ATLAS experiment  (2014-present) \hrule height 0.1mm depth 0.1mm width 165mm}
I joined the ATLAS experiment from 2014 and conducting studies for my Ph.D project. I have been involved in two physics analyses focused on SUSY searches using the early data of LHC Run2 at $\sqrt{s}=13$ TeV, and contributed to publishing the first round results. While based at CERN, I have also been working on the operation of the muon detector, ensuring the smooth and reliable functioning of the system and good data quality. Additionally, I was involved in the upgrade activity of muon detector (New Small Wheel: NSW), developing a databasing system for the MicroMegas detector. \\

\Subsubsection{Operation of Transition Radiation Tracker. (2017-present)}


\Subsubsection{Search for gluinos in the  final state with 1-lepton, jets and missing ET. (2015-2017)} 
Since the discovery of higgs boson at a mass of 125 GeV, there is a dramatically increased interest in the search for gauginos since in typical minimal models, this mass naively implies a scalar top mass of several tens of TeV, beyond the reach of LHC. In the early stage of LHC Run2, the gluino search was particularly well-motivated as its production cross-section increases by a factor 4-40 due to the doubled center of mass energy with respect to Run1. \\

I participated in a gluino search through the final state with exactly one lepton. My major contribution to this analysis is in the background estimation. While careful and dedicated background modeling is crucial for a discovery-oriented analysis, it is challenging in this analysis because the signal region lies in the high energy tails of SM process phase space. The background remaining after the selection are very unusual events, typically associated with many ISR jets. Therefore either MC modeling from first principles or traditional semi-data driven methods which extrapolate kinematics from control regions in the bulk to signal regions in the tails using MC tends to be unreliable. To cope with this, I have been developing a new data-driven estimation method that extrapolates from control regions with two leptons into signal regions with exactly one lepton. As the kinematics are set to be the same between the two regions, the error and potential unknown systematics on the extrapolation are significantly reduced. On the other hand, I have also been investigating the MC modeling of standard model processes (especially regarding top-antitop pair production) in collaboration with the physics modeling group, to understand the large discrepancies with data in extreme phase space, as well as to improve the modeling of the state-of-art particle generators as a first feedback from the 13 TeV data. \\

The analysis has been performed successfully using the Run2 data and two results have been published ([\ref{Pub::ATLPaper_Incl1L_2015}], [\ref{Pub::ATLPaper_Incl1L_2017}] in ``Selected Publication"). No significant deviation from the standard model was observed, and an exclusion limit was set, in a simplified model in which the gluino decays via a chargino, up to a gluino mass of 1.7 $\sim$ 2.0 TeV. \\ \\

\Subsubsection{Search for direct production of EW gauginos. (2014-present)} 
Given that no BSM evidence has so far been observed in Run2, EW-favored split-SUSY scenarios in which the gluino and all other scalar fermions are decoupled become more important. The direct production of electroweak gauginos is the only the probe of such a scenario. \\

I have been committed to this analysis using the 3-lepton final state. I particularly focused on the scenario favored by minimal SUSY models and consistent with dark matter observations. In this scenario, the mass splitting between the lightest neutralino and the second lightest one is typically small, which results in a soft spectrum for the final state particles and missing ET. 
Although three leptons are required for the analysis, it is still tough to distinguish signals against standard model electroweak processes for such signatures. Therefore a sophisticated separation algorithm is needed which exploits the difference in angular distributions and correlations between the transverse mass of particles and so on. Multi-variant likelihoods are a powerful toolkit to cope with such complicated handling of phase space. While adoptive machine learning algorithms such as boosted decision trees (BDT) are the most commonly used in ATLAS, I introduced the method of matrix element likelihood in this analysis, taking the advantage of the fact that the four vectors of the final state particles are well-defined, and that higher order effects have limited influence on the 3-lepton final state. I also improved the integration procedure and succeeded in speeding up the algorithm. Preliminary studies show very promising discrimination power and robustness against experimental and theoretical uncertainty, and further study will follow with more improvements. \\

While these development studies are aiming for a long term analysis with the large datasets expected at the later stage of Run2 and in Run3, I was also involved in the first round Run2 analysis targeting specific models in which the signal cross-section is greatly enhanced by the presence of light sleptons ([\ref{Pub::ATLCONF_EW_SEARCH2016}] in ``Selected Publication"). 
Although these models are phenomenologically optimistic, we drastically extended the exclusion limits from to Run1, up to around 1.1 TeV for the second lightest gaugino mass, and also performed the measurement in the tail of the 3-lepton phase space for the first time in Run2. I contributed to this work extensively, from preparation of dataset, trigger studies, signal region optimization and evaluation of systematical uncertainty.

%\Subsubsection{Operation of ATLAS muon detector (2015-2016)} 

\Subsubsection{Development of a databasing system for the MicroMegas detector (2014-2015)} 
MicroMegas is a sub-detector of the New Small Wheel, aimed for fast and precise measurement of muon tracks in the end-caps. After a successful  R$\&$D and a proposal with TDR, it is now at the stage of fine-tuning detailed design, including the geometry and electronics and so on. Simulation study in accurate geometry therefor became highly important. As my authorship qualification project in ATLAS, I developed a centralized database storing up-to-date geometrical parameters that are sufficient to define whole active area of MicroMegas in simulation.  \\ \\
%%%%%%%%%%%%

%%%%%%%%%%%%%%%%%%%%%%% ILC %%%%%%%%%%%%%%%%%%%%%%%%%%%%
\Subsection{Development of calorimeters for ILC/ILD  (2012-2014) \hrule height 0.1mm depth 0.1mm width 165mm} 
The ILC (International Linear Collider) is a  TeV-scale  electron-positron collider targeting the precise measurement of the higgs boson, and as well as probing new physics in the electroweak sector.  As targeted events typically contain electroweak gauge bosons (W/Z), the precise identification, particularly through their hadronic decay mode, is critical for many of the ILC physics program. The requirement is sub-5\% of energy resolution for a single jet, and accordingly 3-4 GeV resolution for di-jet invariant mass. To achieve this unprecedented benchmark, the ILD (International large detector) is planning to employ the particle flow algorithm (PFA) in which each jet particle is identified and the energy is measured individually. Calorimeters are the key components in this scheme which resolve the showers inside the jet. 
%in which jet particles are classified into charged or neutral particles, and the energy measured by the inner tracker or by the calorimeter is assigned for respective type. This realizes the best resolution and correct scale assignment for each individual particle. Successful association of charged tracks and calorimeter deposits is the key part of the algorithm, even in a dense environment crowded with tens of particles per jet. Therefore a highly granular spatial segmentation is required for the ILD calorimeters. \\
%%%% ??????

During the second half of my master course, I participated in the development of the silicon-tungsten electromagnetic calorimeter (SiW-ECAL) and analogue hadronic calorimeter (AHCAL) in the CALICE Collaboration. \\

\Subsubsection{Commissioning of SiW-ECAL prototypes  (2013)}
To resolve individual clusters inside jets, a highly granular spatial segmentation is required for ILD calorimeters. A silicon-tungsten sampling calorimeter is the one of the leading candidate technology for the ILD electromagnetic calorimeter where the silicon sensor layers are segmented into 5-10mm pixels. The first generation prototype, with about 10000 readout channels, was developed in the CALICE collaboration in 2006, and regularly tested in test beam experiments since then. I analyzed the data collected at a test beam in FNAL in 2011, and verified that the basic performance (linearity, energy resolution, spatial resolution of cluster centeroid etc.) fulfills the requirements ([\ref{Talk:CALICE2013_FNALTB}] in ``Selected Talks at Confereces"). The second-generation prototype was designed as a test for a fully integrated detector including data acquisition system with newly developed readout ASICs (SKIROC2). A number of test beam experiments were performed, in which I participated in setting up data taking systems. ([\ref{Pub:SKIROC2015}] in ``Selected Publication"). \\

\Subsubsection{Study on performance of PFA/SiW-ECAL in a realistic detector setup   (2012-2014)} 
While simulation studies have shown that PFA with the nominal ILD detector model performs very successfully, there are still a number of industrial challenges to achieve the specification of an ideal detector with ideal digitization, for instance the increased dead volume due to realistic PCB thickness or the protection frames around silicon sensor cells (``guard ring") which did not present in the default design.
%For instance, in the ECAL, It is reportedly hard to to manufacture PCBs with sufficient mechanical strength and robustness within the designed thickness (0.5mm), therefore in a realistic detector thicker volume of PCBs has to be taken into account. Also for SiW-ECAL, in order to suppress noise originated from dark current on the surface of silicon sensors, placing a metal rings around a sensor (``guard ring") has been suggested, which however results in dead volume. \\
I performed a set of studies on the impact of such effects due to the features of a realistic detector on physics performance. The jet energy scale and resolution were parametrized as a function of guard ring width,
PCB thickness, fraction of dead pixels/ASICs, noise rate, and cross-talk rate between channels etc., and closely evaluated the dependences via simulation. The results illustrate that the performance of PFA under such conditions is very stable and robust ([\ref{Pub::ECAL_LCWS2013}] in ``Selected Publication"), and also provides primary guidance for the definition of manufacturing requirements in the future. \\



\Subsubsection{Commissioning of the power-pulsing scheme for AHCAL  (2013)} 
Due to the beam structure in ILC (1ms of beam train every 200ms), a power cycling scheme called ``power-pulsing" has been proposed to reduce the power consumption. In this scheme, power supply is synchronized with beams so that power is turned off when collisions do not occur. The DESY FLC group has been working on implementing this into AHCAL electronics. In 2013, I was involved in the commissioning using the test board for my summer student project in DESY. \\
The test board consists of a detector layer instrumented with 30mm $\times$ 30mm square cells of plastic scintillator each equipped with a SiPM, and the readout layer with front-end electronics interfaced to the SiPM attached on the back. Tests were performed with injection of LED pulses, to investigate the physical response of the detector and the readout cycle in a realistic setup. The primary observation when running in the power-pulsing mode was unstable behavior shortly after turning on the electronics, for example worse resolution in single photon spectra from the SiPM and a drop in the gain. I evaluated the impact on data quality and measured the time evolution of the behavior, as well as studying the optimal board configuration for mitigating these effects ([\ref{Talk:CALICE2013_PPHBU}] in ``Selected Talks at Confereces"). \\

\clearpage


%%%%%%%%%%%%%%  BI   %%%%%%%%%%%%%%%%%%%%%%%%%%%%
\Subsection{Testing local-reality in collider experiments.  (2012-2013) \hrule height 0.1mm depth 0.1mm width 165mm}
While quantum mechanics (QM) is without doubt a very successful theoretical framework in describing the microscopic scale of physics, it has suffered from an 
obscure and non-trivial foundation where observables are probabilistic rather than deterministic (``non-realism"), and non-local nature of physics (``non-locality") inevitably arises. 
Such lack of local-realism has been considered a problematic aspect of QM since it was highlighted in the famous ``EPR paradox". However, by testing Bell's inequality, local-realism was experimentally excluded
in 1970-80s using entangled photon pairs. While there were many loopholes in the early experiments, this has been dramatically improved as a result of rapid developments of photon handling and laser technology. \\
%The exclusion of local-realism has become almost complete in the photon pair regime, and nowadays nobody views it as paradox anymore. \\

On the other hand, experiments in systems other than photon pairs are also interesting in terms of testing the universality of quantum nature. In particular, tests with entangled massive particles are non-trivial since they generally behave more classically than photons, as are tests in systems with a high energy scale where particles' Compton wavelength is small and therefore more localized in space. There have been only a few example of such experiments, due to the technical difficulty of preparing and measuring the entangled state. High energy colliders have been suggested for such measurements, since spin-entangled particle pairs can be easily generated via a variety of processes with large statistics, in a wide range of energies from a GeV to an order of 100 GeV, and with various types of interaction. The only problem in this collider regime is that the spin can only be measured through its decay regardless of the observer's free-will, which results in a significant loophole in the interpretation of the experiment. \\
%Ultimate universality of QM is not very trivial. e.g. unitarity problem of BH etc.

There has been little discussion on how much the testing power is reduced by this loophole, and what assumption is  required to make the test complete. My master degree project was devoted to an extensive analysis of this problem. I started by developing a reformulation specifically for the collider setup, and obtained a new very brief representation of Bell's inequality ([\ref{Pub::BI2013_PTEP}] in ``Selected Publication"). I also discussed the feasibility of such measurements in current and future collider experiments, and illustrated that charmonium decays in BES3, the Drell-Yan process in Belle and $H\rightarrow \tau\tau$ in ILC are promising channels for the test. Finally, the size of loopholes was evaluated for each respective channel, and also minimized by imposing consistency with past experiments ([\ref{Talk:ERICE2015}] in ``Selected Talks at Confereces")
%%% originality?????? %%%%



%
%
%%%%%%%%%%%%%%%% Full list of publicatoin %%%%%%%%%%%%%%%%%%%%%%%%%%%%%%%%%%%%%%%%%

%\bibliographystyle{plain}
%\bibliography{publication/nonATLAS}  
%\nocite{*}

%\bibliographystyle{unsrt}
%\bibliography{publication/pub}  


\clearpage

\bibliographystyle{unsrt}
\nocite{*}
\bibliography{publication/pub}  

\if 0
\Section{Full List of Publication \hrule height 0.5mm depth 1mm width 100mm}
\Subsection{Testing local-reality in collider experiments.  (2012-2013) \hrule height 0.1mm depth 0.1mm width 165mm}
%\bibliographystyle{unsrt}    
%\nocite{*}
cite \cite{Chen:2013epa}
\begin{refsection}[publication/testBI]
\printbibliography[heading=subbibliography] % print section bibliography
\end{refsection}

%\Subsection{Development of calorimeters for ILC/ILD  (2012-2014) \hrule height 0.1mm depth 0.1mm width 165mm} 
%\include{fullPub_ILC}
\fi




\end{document}
